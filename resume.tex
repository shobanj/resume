\documentclass[a4paper,12pt]{article}

\usepackage[utf8]{inputenc}
\usepackage[margin=0.5in]{geometry}
\usepackage{lipsum}
\usepackage{tabularx}
\usepackage{wrapfig}
\usepackage{graphicx}
\usepackage{parskip} % For paragraph indentation and spacing
\usepackage{fontspec}
\usepackage{fontawesome5}
\usepackage[hidelinks]{hyperref}
\usepackage[x11names]{xcolor}
\usepackage[absolute]{textpos}

\setmainfont{Garamond}

\graphicspath{ {./images/} }

\hypersetup{
  colorlinks=false,
}

\setlength\itemsep{-0.5em}

\newcommand\cvtitle[2]{\Huge{\bf{#1}} \Huge{#2}}
\newcommand\cvcontact[3] {
  \small {
    \faAt \quad \href{mailto:#1}{#1} \\
    \faPhone \quad \href{tel:#2}{#2} \\
    \faLinkedin \quad {\href{https://www.linkedin.com/in/shobanj/}{#3}}
  }
}

\newcommand\cvhrule{{\color{gray}\hrule}}
\newcommand\cveducation[4]{

  \normalfont\bf{#1, #2}

  \normalfont\small{#3 (#4)}
}

\newcommand\ytl[2]{
  \parbox[b]{3em}{
    \hfill{\color{gray}\bfseries\small{#1}}~$\cdots$~
  }
  \makebox[0pt][c]{$\bullet$}\vrule\quad \parbox[c]{5cm}{
    \vspace{5pt}\raggedright\color{black} \small{#2}.\\[7pt]}\\[-3pt]
}

\newcommand\cvte[2]{\small{\textbf{#1}} ~$\cdots$~ & \small{#2}}

\newcommand\cvsubsection[3]{\subsection*{#1 \quad \small{(#2 \textendash #3)}}}
\newcommand\cvprojectentry[5]{
  \bf{#1}
  
  \normalfont{\it{Client: #2}\normalfont{} \quad (#3 \textendash #4)}
  
}

\begin{document}

\begin{minipage}[b]{.7\linewidth}
  \cvtitle{Shoban}{Jayaraj}
\end{minipage}
\begin{minipage}[b]{.3\linewidth}
  \cvcontact{shobanj@hcltech.com}{+91-98401-46057}{shobanj}
\end{minipage}

\cvhrule

Have been with the IT industry since June '95 with almost thirty years
in architecting and managing multiple projects and functions in
domains like printing, semiconductor, telecom, automation, industrial,
and workflow in addition to focusing on the quality aspects of project
development.

\section*{Professional Experience}

{
\begin{wraptable}{r}{6cm}
  \caption{Timeline}
  \begin{tabular}{l|p{4cm}}
    \hline
    \cvte{2024}{Drove and achieved 150\% Value add and SixSigma initiative targets} \\
    \cvte{2022}{Devloped R / Python based  Knowledge Management Capability assessment and analysis tool - recognized as a SixSigma project \par \href{https://patents.google.com/patent/US20230289515A1}{US20230289515A1} Patent published} \\
    \cvte{2021}{Prototyped 3D / AR app - enabled winning an AR application dev contract} \\
    \cvte{2020}{CMMI Associate - Took part in CMMI assessment and achieved Level 5 for group} \\
    \cvte{2019}{\href{https://patents.google.com/patent/US20190294385A1}{US20190294385A1} - Patent published} \\
    \cvte{2017}{Poster presentation for TUG \textendash `Application of Big Data Analytics in the ATE domain'} \\
    \cvte{2015}{} \\
    \cvte{2014}{SME for OOAD UML, Design Patterns \textendash Trained over 500 members (>250 hrs.) since 2008} \\
    \cvte{2010}{} \\
    \cvte{2001}{ezTest prototyped as a personal initiative and became a line of business} \\
    \cvte{2000}{} \\
    \cvte{1998}{Developed a proxy server for shared internet browsing using Java used withn org} \\
    \cvte{1995}{} \\
  \end{tabular}
\end{wraptable}

\cvsubsection{HCL Technologies, Chennai}{Dec 2002}{Till date}

Excensively managed programs and technically contributed on the Office Automation (Office \& Production Devices), Semiconductor (ATEs, Ball bonders) and Industrial (Frequency Convertors) domains.

\cvprojectentry{Software Development Support}{Leading Semi OEM}{Jan 2024}{Till date}

Group manager managing over 110 members placed at multiple locations
covering QA and SW development for multiple divisions spanning India
(Chennai, Bengaluru), US and Canada. Have been responsible for
handling the operational aspects of running the project covering
forecast, billing, fulfillment and customer connect.

\cvprojectentry{Digital Software Solutions}{US based Office Automation OEM}{Jun 2017}{Dec 2023}

Served as the lead for multiple tracks spanning 45 – 65 members
covering multiple programs including printer drivers, sustenance and
enhamcement of a content management system, web based logs aquisition
and analyze tool for fault diagnostics, LIDAR based app for AI model
training and inference, AR / VR app that used object and state
detection to guide a new user to perform periodic maintenance
activities, enterprise / clould based print solutions, iterations of a
public print kiosk, a playstore equivalent for printers and associated
QA teams. My responsibilities include guiding the team, managing
customer connect, handle contract and commercials, people management
and supported the team on the architectural and design decisions.

Key achievements were – front-ended the Six sigma initiative for the
account by tracking targets, interacting with the respective SPOCs,
coordinating trainings etc., encouraged and filtered ideas for patent
filing and helped the team submit quality ideas for the group and
improved the selection ratio by 70\% (of which I was the primary
applicant for 3 ideas, being finalized for publication with
USPTO). In addition, I also identified and guided on evaluating new
concepts on emerging technologies such as AI / ML, AR (Hands free AR,
Synthetic image generation), NLP etc.,

}

\cvprojectentry{ATE Instruments Program}{US based Office Automation OEM}{Jun 2017}{Dec 2018}

Served as the group manager for the software design and development of
components covering AC, DC, RF instruments and supported the team
technically as well as taking care of customer connect, commercials as
well as team management.
  
Started as a team of 4 members working for a single instrument, was
able to grow this to ~17 members spanning six different instrument and
support projects. Consistently scored a 7/7 on the customer
satisfaction survey for all the projects. We also started the
‘instrument incubation program’ to scale up both new joiners and
lateral hires by focusing on imparting practical knowledge to
instrument development for faster deployment. Was also instrumental in
identifying optimization areas to replicate the knowledge gained in
one instrument and helped in improving productivity on future
products.

\cvprojectentry{64 bit Migration}{US based Office Automation OEM}{Feb 2016}{Jun 2017}

Served as the program manager interacting with the different groups at
the client side to migrate the existing ATE SW application (12M LOC)
from 32 bit to 64 bit to bridge the 4GB RAM limit. Supported in
porting the software from obsolete technology, evaluated 3rd party
components and recommended replacements, conducted performance
evaluation and implemented optimizations while running in the 64 bit
environment, validated all features against the requirements and
ensured conformance to both the 32 bit and 64 bit environment and
support end-user issues during alpha and beta testing and final
release.
 
The program consisted of over 20 members from the software team
interacting with equal number of engineers from the client
organization and SV of similar strength located in three different
geographical locations and time zones. Was responsible for
coordination, tracking of the milestones and progress, identifying and
evaluating risks, productivity measurements and actions, reporting to
different levels of sr. management on the progress and challenges. We
had completed the program successfully 2 weeks ahead of schedule. This
was recognized and awarded as a best project executed at this scale
within the group.

\cvprojectentry{VLT Insight}{Leading Europena Industrial / Power Electronics OEM}{Aug 2011}{Feb 2015}

Served as the architect for designing the PC Software framework, built
to extensible and portable to help the client create new software
application. The framework allowed for a service-oriented-architecture
with C++ as the backend and Adobe AIR as the frontend. The VLT Insight
is one realization of the framework which we designed to manage motor
control devices.
 
The application featured as NUI based GUI with focus on robustness and
performance (faster and able to handle more number of devices
simultaneously compared to the older version). This started as a team
of 4 members when the requirements were analyzed and grew to ~40
members from different functions (Testing, automation, UxD)
collaborating using Agile Scrum. Initially functioned as a technical
manager / architect and as the team grew, primarily focused on the
technical aspects of the project.

\cvprojectentry{Production Print and Office Simulator}{Leading Japanese Printer OEM}{Apr 2009}{Jul 2011}
 
Responsible for developing projects for Konica Minolta Software
Laboratories (the QA department) which aided in improving the
efficiency (man power, cost, time reduction), of the QA people
deployed by Konica Minolta. This included developing simulators (for
Production, Office Printers), test automation framework (to test MIB,
WebServices capabilities of the printer) using C++, Qt on Windows and
porting Windows based simulators to Linux.
 
We have shown an improvement of 17\% over manual testing process and
increased test coverage to 14.7 times of original.
 
The program consisted of multiple teams / projects totaling strength
of 18 to 8 members working off India and Japan. I am responsible for
customer connect, financials and overall team management (estimation,
tracking, performance evaluation, corrective actions, improvements
etc).
 
\cvprojectentry{Digital Development Support}{Semiconductors / ATE}{Jan 2009}{Mar 2009}
 
Providing development and sustenance support for High Speed Memory (Umbra HSM) capabilities scheduled for the release of the latest version of tester software. This consists of two sub-projects involving sustenance covering Timing, MTO and pattern areas targeting the 7.20 ship date and auto-test failure resolution for the 9.80.03 (Umbra Development branch) for the HSM capability. 
 
Am responsible for a team of 7 members, managing the projects and serving as the client interface for the project. 
 
\cvprojectentry{Checkers Maintenance}{Semiconductors / ATE}{Feb 2008}{Dec 2008}

The Test Development Engineering (TDE) Indian team is responsible for
the maintenance programs for the Flex, UltraFLEX, and J750 testers, as
well as developing new test programs and tools.
 
TDE team's responsibility is to assist Teradyne in:

Sustaining Released TDE instrument software.  This includes Checker, PV, Test process, and related tools for DC and AC instruments

Sustaining of the Maintenance Environment; including Maintenance User Interface (MUI), Test Process Manager (TPM) and related tools

SQA

Manage and maintain the J750 Maintenance

New Checker and PV Development 
 
The team has been able to perform well above the stipulated
productivity goals in sustaining the software and has constantly been
revising its goals, the last being a 10\% increase over the last
productivity figures.
 
My responsibility is to manage the team of 11 - 8 members, maintain
customer connect, handle contractual activities like generating SOWs,
billing, monitor the performance of the team and suggest improvement
initiatives.
 
\cvprojectentry{Application Shmoo}{Semiconductors / AT}{Jan 2007}{Nov 2008}

Shmoo plots are graphical charts that display the response of
electrical components or systems by varying a range of conditions and
inputs like voltage, amplitude, current etc. This is used to observe
the operating ranges of a device.
 
Am responsible for managing a team of two to implement the plotting
capability. We are responsible for designing, coding, testing and
verification of the various generations of the application.
 
\cvprojectentry{Analog Tools Program}{Semiconductors / ATE}{Mar 2006}{Nov 2008}

This program involved the team to provide development support and sustenance of all analog tools of Teradyne’s Flex and UltraFlex testers. This involved the team to develop:  
 
A tool (Code Producing Debug Display) to monitor changes made to the various instrument debug displays and provide VBA code generation to allow end users to ‘replay’ the changes at a later point of time. The tool supported over 32 different instruments spanning Flex and UltraFlex. 
A signals and PSet (parameter set) editor that allowed the user to define signals using a GUI and program them to the hardware or re-read them and compare them with an existing set. 
The Characterization Editor that allowed the user to graphically create characterization schematics to test a device, execute the tests and visualize them as a ‘Shmoo’ plot. 
The Tester State Service which provide parametric enumeration support of instrument capabilities which can be queried programmatically. It also provided support for a wizard to generate scaffolding code in DotNet to enable parametric support for new instruments. 
Porting a few capabilities from UltraFlex to Flex which includes: 
A centralized pin viewer which depicts the pin state and is used to provide a holistic view to the test engineer while debugging a test program 
Ability to conduct measurements against a set of limits that were accessible through a programmable interface, but now configurable via the UI. 
 
The development tools were based on various Microsoft technologies
like DotNet (C\#, VB.NET), VB, COM, VC++ and VBA.
 
My responsibility was to manage the specified programs with the team
varying from 3 to 12 engineers, serving as the single point of contact
to the customer, collaborating with other functions like software
verification and test team, automation team and the process compliance
team. I had also contributed technically in most of the projects.
 
Significant achievements:  
Awarded 5/5 on customer satisfaction for the CPDD (DotNet) migration 
Awarded the HCL GoldLine award for successful completion of the CPDD program.  
Was appreciated by the client on my ideas and initiatives of creating reusable components, thus saving time in their future projects 
 
\cvprojectentry{HVD, HDVS and Microwave – Autotest Development and Ownership}{Semiconductors / ATE}{September 2004}{September 2006}

The High Voltage Digital, High Density Voltage Source and Microwave
are instruments that plug into the Teradyne suite of automated test
equipments designed to test devices. The instrument’s features are
exercised by 75+ auto-tests. This project involved us to own all
auto-tests which include development of new auto-tests using VBA,
Perl, C++/COM and CPPUnit to test the features of the instrument,
extending existing tests for cover new features and monitoring and
resolving auto-test failures related to HVD. We had also fixed a few
defects in the HVD component.
 
My responsibility within the team is to plan, schedule and manage the
team of four for development and sustenance of auto-tests. In addition
to management, I am also responsible for process compliance and
contributing to the technical aspects of development.
 
\cvprojectentry{GUI Decoupling}{Semiconductors / Wire bonding}{October 2003}{September 2004}

This project refactors the wire bonder tool by decoupling the domain
specific workflow from UI related activities using the MVC
approach. The goal of the project is to improve developer
productivity, enable split development teams to work on the GUI and
the application and provide a migration path to move to Windows XPE
and .NET from vxWorks and X/Motif. With major feature enhancements
happening in parallel, we chose to use an iterative approach that
incorporates incremental features.
 
My role in this project was to help recommend the architecture,
estimate, plan, architect and collaborate with the various teams in
realizing the design. I also managed a team of four to complete the
code communication framework using C++, Visual C++, vxWorks.
 
\cvprojectentry{Distributed Material Handling System}{Semiconductors / Wire bonding}{January 2003}{October 2003}

The Distributed Material handling system is a new feature request to
the wire bonder suite of tools developed by Kulicke and Soffa
Industries that aims to move away from the proprietary VME bus to off
the shelf PC based systems via Ethernet. It allows multiple
distributed modules that can be connected to the bonder tool. Each
module is responsible for controlling a specific set of devices like
sensors, steppers and solenoids. The project also implemented a custom
distributed shared memory component to allow the components to talk
with each other.
 
The project was developed using vxWorks on the x86 platform with the
target being a stepper controller running on the Blackfin series of
DSP processors by Analog devices. The stepper board was connected to
the sensor and solenoids via a SPI bus. The project involved
implementing a messaging layer between the host and the target in C++
and a distributed shared memory component that was used by the host
and the target to send stepper commands.
 
Was responsible for architecting the host part of the system using
OOAD, Rational Rose and UML. Initially, the messaging layer was built
on VC++ using POSIX and was later migrated to vxWorks. I managed a
team of four members to implement the host part of the software. I was
also responsible for providing on-site training and transition to the
members of the Singapore KnS division

\cvsubsection{Think Business Networks, Coimbatore}{Aug 1997}{Dec 2002}

\cvprojectentry{Digital Nervous System}{Process Workflow}{September 2002}{December 2002}
 
Managed the DNS project, a workflow automation framework that allows
IT related processes to be automated. It aims at reducing the
productivity dip that is caused by following the manual process by
handholding the users on what needs to be done and how. Users can
interactively model processes via activity diagrams, deploy them for
instantiation, map the processes to the various KPAs defined by SEI
CMM (for internal assessment), extract quality metrics etc.
 
My responsibilities were to manage a team of eight members and
incorporate functionalities like CVS support for document archival,
conditional branching and joining of activities within processes etc.
 
DNS was implemented on the .NET platform. 
 
\cvprojectentry{Test Automation Layer}{Test Automation}{August 2002}{December 2002}

 
Functioned as an architect to design and validate the Test Automation
Layer (TAL) that allowed producing faster automation solutions by
providing functional automation management and deployment tools on top
of existing automation frameworks. TAL employs various test automation
methodologies like CSDDT (Continuous Synchronized Data Driven
Testing), Action / keyword driven testing to provide effective test
automation solutions.
 
\cvprojectentry{Real-time Fault Receiver}{Telecom / Networking}{Mar 2002}{July 2002}

 
Designed the system and managed a team of four for the project:
`Real-time Fault Receiver', a component add-on to our product, `Think
EMS' to provide the capability to minimize SNMP based trap loss by
providing a real-time solution using RTJ specs.
 
This highly configurable system functions as a real-time component
running on TimeSys Linux/GPL with Real-time Java extensions to capture
UDP based SNMP traps and relay them back to the EMS for event
correlation and consolidation. The RFR implements its own high
performance customized persistence engine and provides a 20x
performance boost in handling SNMP TRAPS compared to the current EMS
implementation.
 
\cvprojectentry{EzTest}{Internet / Web}{July 2001}{September 2002}

 
Was responsible for the architecture and management of the EzTest
project, a part of the Enterprise / University Information Portal
project (E/UIP). EzTest features automated evaluation of candidates on
selected skill sets aiding in recruitment and training
assessment. EzTest used open-sourced software to reduce the cost of
the product.
 
The assessment engine was marketed towards the educational, corporate
and IT enabled services (call centers) segments. Incidentally, the
assessment engine was started by myself as a personal initiative to
help us with our recruitment efforts and was eventually adopted by our
organization. The product was completed with a team of three members.
 
\cvprojectentry{Nuera EMS}{Telecom / Networking}{June 2000}{July 2001}

 
Was one of the key members involved in the design and development of
an EMS for managing and monitoring Remote Digital Terminals (RDT)
manufactured by Nuera Communication Inc, CA, USA, for their customers
AT\&T. The EMS was designed to manage Fault, Configuration, Performance
and Security (FCAPS) of the system. The RDT was responsible for
enabling IP based voice and data communication within cable TV
networks and interfaces with the external PSTN.
 
Was responsible for the design and managed a team of three to
implement the Fault, Performance and Diagnostics servers for the
system. The servers were based on a pluggable architecture supporting
SNMP southbound and with northbound capabilities to talk to
higher-level managers (like HP OpenView) via SNMP, RMI or e-mail.
 
\cvprojectentry{zGenie}{Internet / Web}{October 1999}{October 2000}

Zgenie is a web community based information sharing tool that brings
content to your desktop rather than wade through millions of
non-relevant information available in the web.
 
Was one of the key architects of the system and managed the client
side implementation of the application with a team of four. The client
side implementation consisted of building COM based plug-in to
Internet Explorer, a lightweight proxy server and a binary
interception engine that allows the users to `Alt+Click' on any text
in any GUI based application and use it as a base to pull information
off the web. The application also supported the capability to update
itself over the web. The product entered beta with 300+ users using
the application. The product was developed with Microsoft Visual C++
6, with MFC and COM.
 
The application suite was used to power our knowledge management tool
`ezInfo' on the client side.
 
\cvprojectentry{WorldStreet Sales}{Securities / Financial}{August 1998}{September 1999}

 
WorldStreet Sales is a financial portal for the securities industry,
capable of consolidating external information like stock quotes, news
feeds, research articles from various sources like IDC, Reuters in
real-time, there by allowing sales persons to provide timely advice to
their clients.
 
My job, as a consultant was to analyze the performance of the system,
identify the shortcomings and overcome the performance issues with
better solutions. This involved developing solutions that required
parallel generation of presentation data for portlets, dynamic caches
to manage real-time quotes and database optimizations.
 
Was responsible for bringing back two projects to be developed
offshore, one being a chart server capable of plotting financial
charts using pricing information obtained from data-feeds maintained
by Reuters and IDC. The other is responsible for uploading external
IDC data-feeds into our database.
 
These projects made extensive usage of Java, RMI, JNI, and JDBC to
achieve its goals.
 
\cvprojectentry{NetExpress}{Telecom / Networking}{August 1997}{August 1998}

 
NetExpress is a 3-tired distributed system on Windows NT, for the
management of complex services on Nortel’s DMS 100 switches. Basically
the system is responsible for service provisioning (Service Management
System) in the DMS 100 switches and also acts as Operations Support
Systems for the service providers. The system consists of thin clients
spread across a WAN, being serviced by our set of server applications
running on NT server with a backend ORACLE database. This system was
developed entirely in Visual C++ involving rule based, network
modeling and limited object database implementation.
 
Ceon (formally known as FirstTel systems) developed NetExpress for
some of the largest Telcos in the world (Telstra, British Telecom and
etc.).
 
Developed ASUBS (Automatic Service Unit Billing System), a billing
system for the features (around 190 features) provided to customers in
the DMS 100 switch. There was no proper system to produce the billing
for the DMS 100 switches and this is the very first system that
generated accurate feature based billing for DMS 100 switches. ASUBS
is a rule-based system that features capabilities to add new billing
requirements and modify the existing rule without changing the
application.
 
Was also responsible for managing a team of seven for a feature point
release of NetExpress.


\cvsubsection{L-Cube Innovative Solutions, Chennai}{Jun 1995}{Aug 1997}

\cvprojectentry{Electronic Service Manual}{Document Publishing}{Jun 1995}{Aug 1997}

 
The Electronic Service Manual (ESM) helps in providing quick access of
digitized documents to service personnel in troubleshooting complex
machineries. It featured network decision trees to aid the personnel
in zeroing on the problem, dynamic forms to report problems, an
internal scripting framework, with support for rendering various types
of documents like RFT, WordPerfect Documents, SGML and HTML. The ESM
works as a standalone application and is also capable of collaborating
with other users over the network. The latest incarnation was capable
of generating Internet ready manuals dynamically for access over the
web. The ESM was developed using C++.
 
I functioned as a software engineer, responsible for implementing a
display renderer capable of rendering complex documents with supports
for tables, columns etc. Was also responsible for developing the
web-rendering engine that exports regular word processing documents as
HTML to be accessible over the net.
 
I was also responsible for building the entire automated test
framework, based on MS Test, an early version of Rational Visual Test.

\section*{Education}

\cveducation{M.S.}{Software Systems}{BITS Pilani}{CGPA 7.94}
\cveducation{B.E.}{Computer Science}{Barathiar University}{74.77\%}

\section*{Other Achievements}

\lipsum

\begin{itemize}
  \item{\bf{HCL Technologies}}
    
    \begin{itemize}
      \setlength\itemsep{-0.5em}
      \item Authored the paper `Application of Big Data Analytics in
        the ATE domain' which was selected for poster presentation as
        part of the ATE user meet up (2017)
      \item Published a paper on `Big data analytics' in the internal
        newsletter

      \item Created proof of concept application on auto-test
        regression analysis using the ELK (Elasticsearch, Logstash and
        Kibana) stack which led to the focus of extending the same
        within the group for different projects for analysis.

      \item Best project award (2) for execution and completion of
        projects within the semi group (2016 – 2017)

      \item Awarded the HCL Goldline award (2) for a 5/5 Customer
        Satisfaction rating and the successful completion of the CPDD
        program (2012)

      \item A paper on `Machine Learning' was shortlisted as part of
        the internal paper presentation contest within the group

      \item TechCeed Certified Trainer (HCL Training Department) for
        OOAD / UML, Use Cases, Design Patterns

      \item Recognized as an Employee First Leader based on 360 degree
        feedback within HCL and listed in the top 10\% of all managers

      \item Serving as the Subject Matter Expert (SME) for OOAD and
        UML since 2008 at HCL

      \item Awarded by the Talent Transformation Group (training
        department at HCL) for imparting more than 250 hours of
        training for over 500 members on OOAD with UML.

      \item Served as SKO (Senior Knowledge Officer, HCL, Teradyne
        IDC, Instrumentation Group)

    \end{itemize}

  \item{\textbf{Think Business Networks}}

    \begin{itemize}
      \setlength\itemsep{-0.5em}
      \item Was part of the SEI CMM internal audit team 

      \item Served as a member of the Quality Improvement Team 

      \item Responsible for `Open Source Software' awareness which has
        lead the organization to use both open source software as well
        as to contribute to the movement

      \item Served as the CTO, conducted many technology based team
        activities within the organization to create awareness of KM
        practices, reusability etc.

      \item Developed, as a personal initiative, EzTest an automated
        testing system to help us with recruitment activities. EzTest
        was adopted by the organization and was commercialized,
        targeting all universities, recruitment agencies, ITeS
        industries, training institutes and schools. EzTest was
        awarded the Quality Improvement Team's best corrective action
        project for 2001
        
    \end{itemize}
\end{itemize}

\section*{Hobbies}

\end{document}
